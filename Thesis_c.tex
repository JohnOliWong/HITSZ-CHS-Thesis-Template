%!TEX program = xelatex

\documentclass[14pt]{article}
% Chinese
\usepackage[UTF8]{ctex}
% mathematical equations
\usepackage{amsmath}
\usepackage{amssymb}
% graphics
\usepackage{graphicx}
% geometry
\usepackage{geometry}
\geometry{a4paper, margin=1in}
% tables
\usepackage{booktabs}

% running title and page numbering
\usepackage{fancyhdr}
\pagestyle{fancy}
% clear existing format
\fancyhf{}
\fancyhead[C]{哈尔滨工业大学课程设计论文}
\fancyfoot[C]{\thepage}
\renewcommand{\headrulewidth}{0.4pt}
\renewcommand{\footrulewidth}{0pt}

% font, font size and spacing
\usepackage{fontspec}
\usepackage{setspace}
\setmainfont{Times New Roman}
\setsansfont{Times New Roman}
\setmonofont{Times New Roman}
\setCJKmainfont{SimSun}
\renewcommand\normalsize{
    \fontsize{12pt}{17.5pt}
    \selectfont
}
\setstretch{1.25}

% title format
\usepackage{titlesec}
% 小二-18 三号-16 小三-15 四号-14 小四(正文)-12 五号-10.5
\titleformat{\section}
    {\centering\heiti\zihao{-2}}
    {第\chinese{section}章}
    {0.1cm}
    {}
\titleformat{\subsection}
    {\heiti\zihao{-3}}
    {\arabic{section}.\arabic{subsection}}
    {0.1cm}
    {}
\titleformat{\subsubsection}
    {\heiti\zihao{4}}
    {\arabic{section}.\arabic{subsection}.\arabic{subsubsection}}
    {0.1cm}
    {}
\titleformat{\subsubsubsection}
    {\heiti\zihao{-4}}
    {\arabic{section}.\arabic{subsection}.\arabic{subsubsection}.\arabic{subsubsubsection}}
    {0.1cm}
    {}

\begin{document}
% title page
\pagenumbering{gobble}
\begin{titlepage}
    \begin{center}
        \includegraphics[width=0.5\textwidth]{hitsz.png}\\
        \vspace{8cm}
        {\heiti\zihao{-2}文章标题}\\
        \vspace{1.5cm}
        {\heiti\zihao{-2}作者姓名}\\
        \vspace{6cm}
        {\zihao{-2}哈尔滨工业大学}\\
        \zihao{-2}2024年11月
    \end{center}
\end{titlepage}

% table of contents
\newpage
\pagenumbering{Roman}
\tableofcontents

% Chinese abstract
\newpage
\section*{摘要}
\par
本文主要研究内容如下。\\
\noindent {\heiti 关键词:}关键词

% English abstract
\newpage
\section*{Abstract}
\par
This is the abstract.\\
\noindent \textbf{Keywords:}keyword

% Chapter 1
\newpage
\pagenumbering{arabic}
\section{绪论}
\subsection{课题背景及研究的目的和意义}
\subsubsection{课题背景}
本文主要研究内容如图1-1所示。
\begin{center}
    \includegraphics[width=0.5\textwidth]{hitsz.png}\\
    \zihao{5} 图1-1 系统结构
\end{center}
\par
多头注意力机制的输出如式(1-1)所示。
\begin{align}
    Attention=Softmax(\frac{QK^T}{\sqrt{d_k}})V \tag{1-1}
\end{align}
式中,$Q,K,V \text{ —— 队列,键,值向量}$;
\par
\hspace{0.2cm} $d_k \text{ —— 值向量维度}$
\par
本文取得的准确率如表1-1所示。
\begin{center}
    \zihao{5} 表1-1 测试准确率\\
    \begin{tabular}{ccc}
        \toprule
        模型 & 训练准确率 & kappa\\
        \midrule
        CSP & 80\% & 0.6\\
        EEGConformer & 90\% & 0.7\\
        \bottomrule
    \end{tabular}
\end{center}
\subsubsection{研究的目的和意义}
最经典的脑电分类算法是Koles在1991年提出的共空间模式算法(Common Spatial Pattern, CSP)。$^{[1]}$

\newpage
\section*{参考文献}
\addcontentsline{toc}{section}{参考文献}

\end{document}